\section{Introduction} \label{section:introduction}
This report corresponds to the System Requirements Review and Conceptual Design Review for the Spaceshot project of the Purdue Space Program High Altitude team. The purpose of this review is twofold: first, we seek feedback on our system requirements; we have a set of stakeholder requirements, which we have flowed down to the level of each vehicle subsystem,  and we believe these requirements are complete and sufficient to define our project. Second, we present our conceptual design of the vehicle, and show that it is capable of meeting the requirements we have specified.

We believe the report is structured logically to achieve those objectives. In this section we offer a short introduction to the team and its finances. Next, \Cref{section:requirements} covers the stakeholder and functional requirements, and flows them down to the level of each subsystem. \Cref{section:sizing} begins discussion of the vehicle design with the initial sizing process, and our goals for future simulation. The propulsion system is next, in \Cref{section:propulsion}, followed by avionics in \Cref{section:avionics}, mechanisms in \Cref{section:mechanisms}, and structures in \Cref{section:structures}. Each of these component sections discusses the motivating requirements and the planned implementation that will satisfy them. Finally, \Cref{section:nextsteps} discusses our next steps with the project after this review, including aspects of the project we consider highest risk.

\subsection{Purdue Space Program High Altitude}
High Altitude (HA) is a project team within Purdue Space Program (PSP) that was formed in May 2021. High Altitude’s objective is to fly a two stage, student developed rocket to the K\'{a}rm\'{a}n Line: 100 kilometers above mean sea level. The team was formed with the experience and leadership from the now-defunct PSP Solids team, which competed annually in the Spaceport America Cup. Over the course of the past year, High Altitude has continued to develop skills across the team in design iterations and flights as the team continues to move into more detailed work on the spaceshot rocket.

Since its formation, High Altitude has been involved in rapid iteration and prototyping of many smaller-scale rockets. Last year, the team conducted three launches that began developing experience for our team. This started with an initial L2 kit rocket, the Wildman Darkstar Extreme, and its launch in September 2021. The team’s next launch was in December; it was fully designed and constructed by our team and made primarily out of carbon fiber. The third and most recent launch was a reflight of the Darkstar. After these launches, the team began work on the spaceshot project; this Design Review will conclude the first phase of that work.

\subsection{Budget}
The High Altitude team receives funding each semester from Purdue organizations including Purdue Engineering Student Council (PESC) and the Purdue Engineering President’s Council (PEPC).  These merit funds total up to \$6,000 per semester.  The team launched a successful crowdfunding campaign in the Spring of 2022 to raise over \$3,000 and also participates in fundraising events through Purdue Athletics.  In addition, we have applied for scientific research grants through organizations such as NASA to support the project’s development.  These research grants are limited by the type of research being completed by HA as there is not an experimental payload included onboard the rockets.

These funds are reallocated each semester to each technical team based on the current projects of each team.  Currently, HA has about \$8,000 with an expected addition of \$3,000 from the PESC Merit Fund before the end of the year.  For the Spring 2023 semester, the Avionics team will receive \$2,000 for the research and development of a flight computer as well as the purchase of a commercial avionics board to be tested on an L1 kit rocket.  The Mechanisms team will receive \$500 to construct and test the de-spin mechanism as well as the recovery system.  The Propulsion team will be designing and building a test stand at Zucrow Laboratories to characterize solid rocket motors; the cost of this project is dependent upon the involvement of other research groups.  Structures will continue to finalize the design for spaceshot; prototyping, manufacturing, and testing the airframe is estimated to cost between \$4,000 - \$10,000 which will be allocated  incrementally during the next few semesters.  Future budgeting will involve attempting to obtain funding from companies, institutions, and foundations.
